\seccion{Esperanza, momentos y funciones generadoras}
\label{s:esperanzamomento}


%%%%%%%%%%%%%%%%%%%%%%%%%%%%%%%%%%%%%%%%%%%%%%%%%%%%%%%%%%%%%%%%%%%%%%%%%%

\emph{introducci\'on...}

%%%%%%%%%%%%%%%%%%%%%%%%%%%%%%%%%%%%%%%%%%%%%%%%%%%%%%%%%%%%%%%%%%%%%%%%%%
\subseccion{Momentos de una distribuci\'on}


Una variable aleatoria continua $X$ tiene asociado un \emph{promedio} o \emph{media} (tambi�n llamado \emph{valor esperado o de expectaci�n}) $\langle x\rangle$ que se obtiene pesando cada valor de $x$ con la probabilidad asociada a ese valor, $p(x)\,dx$, e integrando sobre el rango permitido de $x$: 
$$
\mu = E[X] = \langle x\rangle = \int_{\Omega} x \ p(x)\,dx
$$
si la integral existe. La \emph{esperanza} de la variable aleatoria $X$ representa el valor medio que puede tomar, entre todos los eventos de una prueba. 



...........

\hfill

En el caso de una variable aleatoria discreta $X$ que toma valores en $\Omega=\{x_1, \ldots, x_N\}$, la esperanza de la variable viene dada por 
$$
E[X] = \sum_{n=1}^N x_n \, p(x_n) . 
$$

....

....

\vspace{1.5pt}
%%%%%%%%%%%%%%%%%%%%%%%%%%%%%%%%%%%%%%%%%%%%%%%%%%%%%%%%%%%%%%%%%%%%%%%%%%
\subsection{Funciones generatrices}
%%%%%%%%%%%%%%%%%%%%%%%%%%%%%%%%%%%%%%%%%%%%%%%%%%%%%%%%%%%%%%%%%%%%%%%%%%

....

\vspace{1.5pt}
%%%%%%%%%%%%%%%%%%%%%%%%%%%%%%%%%%%%%%%%%%%%%%%%%%%%%%%%%%%%%%%%%%%%%%%%%%
%%%%%%%%%%%%%%%%%%%%%%%%%%%%%%%%%%%%%%%%%%%%%%%%%%%%%%%%%%%%%%%%%%%%%%%%%%