Obtuvo el t�tulo de Licenciada en F�sica en la Facultad de Ciencias Exactas de la  Universidad Nacional de La Plata, y el grado de Doctora en F�sica en la misma casa de altos estudios. Es Investigador Independiente del Consejo Nacional de Investigaciones Cient�ficas y T�cnicas, con lugar de trabajo en el Instituto de F�sica La Plata. Su especialidad es la teor�a y geometr�a de la informaci�n en mec�nica cu�ntica. Posee cargo docente de Profesor Adjunto en el Departamento de Matem�tica de la Facultad de Ciencias Exactas de la UNLP, desempe��ndose 
desde 2013 %%
como integrante del Equipo Coordinador de la asignatura An�lisis Matem�tico~II (CiBEx). 
%%junto a la Dra Mar�a Teresa Mart�n (2013--2015) y a la Dra Mar�a Eugenia Garc�a (desde 2016). En colaboraci�n con /  Junto a 
%los coautores de esta obra, han impartido el curso avanzado ``M�todos de geometr�a diferencial en teor�a de la informaci�n'' en la Facultad de Ciencias Exactas de la UNLP y en la Facultad de Matem�tica, Astronom�a, F�sica y Computaci�n de la Universidad Nacional de C�rdoba.
Ha impartido cursos de grado avanzados y de posgrado en la Facultad de Ciencias Exactas de la UNLP y en la Facultad de Matem�tica, Astronom�a, F�sica y Computaci�n de la Universidad Nacional de C�rdoba. 
Tambi�n ha participado en el dictado del curso de grado ``Probabilidades'' como Profesor Visitante de la Universit� Grenoble--Alpes en Francia. 
